%%%%%%%%%%%%%%%%%%%%%%%%%%%%%%%%%%%%%%%%%
% Twenty Seconds Resume/CV
% LaTeX Template
% Version 1.1 (8/1/17)
%
% This template has been downloaded from:
% http://www.LaTeXTemplates.com
%
% Original author:
% Carmine Spagnuolo (cspagnuolo@unisa.it) with major modifications by
% Vel (vel@LaTeXTemplates.com)
%
% License:
% The MIT License (see included LICENSE file)
%
%%%%%%%%%%%%%%%%%%%%%%%%%%%%%%%%%%%%%%%%%

\documentclass[a4paper]{twentysecondcv}


\profilepic{} % Profile picture

\cvname{Juha Autero} % Your name
\cvjobtitle{Senior Software Engineer} % Job title/career

\cvdate{2 May 1974} % Date of birth
\cvaddress{Meteorinkatu 6 B 23 \newline 02210 Espoo} % Short address/location, use \newline if more than 1 line is required
\cvnumberphone{+358 40 742 9483} % Phone number
\cvsite{http://jautero.iki.fi} % Personal website
\cvmail{jautero@iki.fi} % Email address

%----------------------------------------------------------------------------------------

\begin{document}

\aboutme{}

% Skill bar section, each skill must have a value between 0 an 6 (float)
\skills{}


% Skill text section, each skill must have a value between 0 an 6
\skillstext{}

%----------------------------------------------------------------------------------------

\makeprofile % Print the sidebar

\section{Interests}

Active member of science fiction societies in Helsinki capital area:
\emph{Helsingin yliopiston science fiction klubi},
\emph{Helsingin science fiction seura},
\emph{Espoon science fiction- ja fantasiaseura}. Member of organizing committee
in \emph{Finncon 2006} and \emph{Finncon 2009}. Member of
\emph{World Science Fiction Convention} in 2005, 2014, 2015, 2016, 2017 and 2018.
Programming projects in GitHub: \url{http://www.github.com/jautero}.
Twitter: \url{http://twitter.com/jautero/} Maintainer for \texttt{sfnet.atk.sao}
FAQ. I started computing with \emph{Amstrad CPC464} 8-bit computer in 1986.
My next computer was XT clone \emph{Amstrad PC1640} few years later. In 1995 I
got a \emph{Compaq} with 486 processor and \emph{Windows 3.1}, which I few years
later switched to \emph{SuSE Linux}. I got introduced to \emph{Debian} when
starting working at \emph{F-Secure} and later started to use it at home. That
lasted until I bought my first \emph{Mac} in 2006. Member of Helsinki Hacklab
\url{http://helsinki.hacklab.fi} where I'm currently learning programming FPGAs.


\section{Experience}

\begin{twenty} % Environment for a list with descriptions
	%\twentyitem{}{<title>}{<location>}{<description>}
\end{twenty}

\section{Education}

\begin{twenty} % Environment for a list with descriptions
	%\twentyitem{<dates>}{<title>}{<location>}{<description>}
\end{twenty}

\end{document}
